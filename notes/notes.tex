\documentclass[a4paper,12pt]{article}

\usepackage[english]{babel}
\usepackage[utf8]{inputenc}
\usepackage[OT1]{fontenc}
\usepackage{mathtools}
\usepackage{amsmath}
\usepackage{xcolor}

\newcommand{\pder}[2]{\frac{\partial #1}{\partial #2}}
\newcommand{\A}{\mathbf{A}}
\newcommand{\C}{\mathbf{C}}
\newcommand{\D}{\mathbf{D}}
\newcommand{\E}{\mathbf{E}}
\newcommand{\F}{\mathbf{F}}
\renewcommand{\H}{\mathbf{H}}  % ¶
\renewcommand{\P}{\mathbf{P}}  % ¶
\renewcommand{\S}{\mathbf{S}}  % §
\newcommand{\Q}{\mathbf{Q}}
\newcommand{\Div}{\text{Div}}
\newcommand{\Eps}{\mathcal{E}}
\newcommand{\0}{\mathbf{0}}


\title{Notes on Thermo-Electro-Visco-Elasticity modelling}
\author{Rogelio Ortigosa, Jesús Martínez, Miguel Masó}
\date{September 5, 2025}

\begin{document}

\maketitle

Estas notas corresponden a los borradores desarrollados en las fechas del 10 al 14 de noviembre durante una estancia en la UPCT.


\section{Thermo-mechanical energy}

\begin{equation*}
    \Psi(\F, \E_0, \theta) = \bar{\Psi}_R(J) - \bar{g}(\theta) \theta_R \bar{\eta}_R(J) +
    (1 + \hat{g}(\theta))
    \left(
        \hat{\Psi}_R(\F,\E_0) + \Psi_\alpha(\C_{e\alpha})
    \right)
\end{equation*}
with
\begin{align*}
  \bar{g}(\theta) &= \frac{1}{\bar\gamma+1}\left(\left(\frac{\theta}{\theta_R}\right)^{\bar\gamma+1}-1\right), \\
  \hat{g}(\theta) &= \frac{1}{\hat\gamma+1}\left(\left(\frac{\theta}{\theta_R}\right)^{\hat\gamma+1}-1\right), \\
  \bar\eta_R(J)   &= \alpha(J - 1) + \frac{c_v^0}{\bar\gamma}.
\end{align*}

\begin{align*}
  \P   = \frac{\partial \Psi}{\partial \F}, \quad
  \D_0 = -\frac{\partial \Psi}{\partial \E_0}, \quad
  \eta = -\frac{\partial \Psi}{\partial \theta}, \quad
  c_v  = -\frac{\partial^2 \Psi}{\partial \theta^2}.
\end{align*}



\textcolor{gray}{
\noindent Note: in order to satisfy the third law of thermodynamics, the entropy must vanish at rest.
}




\section{Conservation laws}

\paragraph{Linear momentum}
\begin{equation*}
    \int_\Omega \mathbf{P}(\F, \E_0, \theta) : \nabla v_u \ d\Omega - \int_\Omega \mathbf{f}_0 \cdot v_u \ d\Omega = 0.
\end{equation*}

\paragraph{Gauss law}
\begin{equation*}
    \int_\Omega \D_0(\F, \E_0, \theta) \cdot \nabla v_\varphi \ d\Omega = 0,
\end{equation*}

\paragraph{Energy conservation}
\begin{equation*}
    \frac{d}{dt}(\theta\eta) - \eta\dot\theta - \D_\text{vis}
    + \Div\Q - r_R = 0.
\end{equation*}
with the corresponding weak form
\begin{equation*}
    \int_\Omega \left(\frac{d}{dt}(\theta\eta) -\eta\dot\theta -\D_\text{vis}\right)v_\theta d\Omega
    + \int_\Omega \kappa\nabla\theta\cdot\nabla v_\theta d\Omega
    - \int_\Gamma r_R v_\theta d\Gamma = 0.
\end{equation*}


\begin{equation*}
    \rightarrow K = J^{-1}\H^T\kappa\H
\end{equation*}




\section{Semi-discrete blance of total energy}

Taking the test functions $\{\mathbf{v}_u, v_\phi, v_\theta \}$ as $\{\Delta\mathbf{u}/\Delta t, \Delta\phi/\Delta t, 1\}$

\begin{equation*}
    \mathcal{W}_{u_{n+1/2}}^\text{int} = \frac{1}{\Delta t}\int_\Omega \P_h:\Delta\F dV
\end{equation*}
\begin{equation*}
    \mathcal{W}_{\phi_{n+1/2}}^\text{int} = \frac{1}{\Delta t}\int_\Omega \D_{0,h}\cdot\Delta\D_0 dV
\end{equation*}
\begin{equation*}
    \mathcal{W}_{\theta_{n+1/2}}^\text{int} = \frac{1}{\Delta t}\int_\Omega (\Delta(\eta\theta) + \eta_h\Delta\theta) dV
\end{equation*}

In the absence of thermal contributions, the algorithmic internal energy balance reduces to
\begin{equation*}
    \mathcal{W}_{n+1/2}^\text{int} = \frac{\Delta\Psi}{\Delta t}
\end{equation*}




\section{Numerical example}

\subsection{Geometry}

$$
\Omega = 100 \times 100 \times 1
$$


\subsection{Constitutive model}

\paragraph{Neo-hookean parameters}
\begin{align*}
\mu_1   &= 1.0 \\
\mu_2   &= 1.0 \\
\lambda &= 10.0
\end{align*}

\paragraph{Ideal dielectric parameters}
\begin{equation*}
\epsilon = 1.0 
\end{equation*}


\paragraph{Thermal parameters}
\begin{align*}
c_v^0      &= 17.385  \\
\theta_R   &= 293.15  \\
k          &= \lambda + 2(\mu_1+\mu_2) \\
\alpha     &= 22.33k\times10^{-5} \\
\bar\gamma &= 1.0 \\
\hat\gamma &= 1.0
\end{align*}


\subsection{Dirichlet boundary conditions}

\begin{align*}
\mathbf{u} &= \mathbf{0} \ , \quad \text{at} \ \Gamma_{u} \\
\phi    &= 0 \ , \quad \text{at} \ \Gamma_{\phi,\text{bottom}} \\
\phi    &= \phi_D \ , \quad \text{at} \ \Gamma_{\phi,\text{top}}
\end{align*}
\begin{align*}
\phi_D = 0.3\sin\left(\frac{\pi t}{2}\right)
\end{align*}


\subsection{Neumann boundary conditions}

\begin{align*}
r = r_N \ , \quad \text{at} \ \Gamma_{\theta}
\end{align*}
\begin{align*}
r_N = \frac{500}{\pi 15^2}\sin\left(\frac{\pi t}{2}\right)
\end{align*}


\subsection{Results}

$$
\eta_0 = \int_\Omega\frac{c_v^0}{\bar\gamma}dV = 173850.0
$$




\section{Patch test}

\subsection{Geometry}

$$
\Omega = 0.1 \times 0.1 \times 0.1
$$

\begin{align*}
  u_z  = 0    \ &, \quad \text{at} \ \Gamma_\text{bottom} \\
  \phi = 0    \ &, \quad \text{at} \ \Gamma_\text{bottom} \\
  \phi = \alpha t \ &, \quad \text{at} \ \Gamma_\text{top} \\
\end{align*}



\subsection{Energy balance}

\begin{equation*}
    \frac{d}{dt}(\eta\theta) - \eta\dot\theta - \D_\text{vis}
    + \Div\Q - r_R = 0.
\end{equation*}

\begin{equation*}
    \dot\eta = 0.
\end{equation*}


$$
m \, \alpha \, \text{cof}(\C_{e_\alpha}) \otimes \text{cof}(\C_{e_\alpha})
$$

$$
\alpha = \text{tr}\left(\frac{\partial \S_{e_\alpha}'}{\partial \C_{e_\alpha}}\right)
$$


\end{document}
