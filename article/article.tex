\documentclass[a4paper,12pt]{article}

\usepackage[english]{babel}
\usepackage[utf8]{inputenc}
\usepackage[OT1]{fontenc}
\usepackage{mathtools}
\usepackage{amsmath}
\usepackage{xcolor}
\usepackage{nccmath}  % Flush left equations: fleqn

\newcommand{\pder}[2]{\frac{\partial #1}{\partial #2}}
\newcommand{\A}{\mathcal{A}}
\newcommand{\C}{\mathbf{C}}
\newcommand{\D}{\mathbf{D}}
\newcommand{\E}{\mathbf{E}}
\newcommand{\F}{\mathbf{F}}
\newcommand{\G}{\mathcal{G}}
\newcommand{\I}{\mathbf{I}}
\renewcommand{\H}{\mathbf{H}}  % ˝
\renewcommand{\P}{\mathbf{P}}  % ¶
\renewcommand{\S}{\mathbf{S}}  % §
\newcommand{\Q}{\mathbf{Q}}
\newcommand{\Div}{\text{Div}}
\newcommand{\Eps}{\mathcal{E}}
\newcommand{\0}{\mathbf{0}}


\title{Notes on Thermo-Electro-Visco-Elasticity modelling}
\author{Rogelio Ortigosa, Jesús Martínez, Miguel Masó}
\date{September 5, 2025}

\begin{document}



\section*{Isochoric Energy Formulation}

Based on the Third Law of Thermodynamics and the multiplicative decomposition of the specific heat, the isochoric specific heat coefficient is defined as:
\begin{equation}
    \hat{c}_v(\hat{\F}, \theta) = \frac{1}{\xi_R} \hat{\eta}_R(\hat{\F}) g(\theta),
\end{equation}
where $\xi_R$ is a normalization constant defined by the integral:
\begin{equation}
    \xi_R = \int_0^{\theta_R} \frac{g(\vartheta)}{\vartheta} \, d\vartheta.
\end{equation}

By integrating the thermodynamic relation $\hat{c}_v = \theta (\partial \hat{\eta} / \partial \theta)$ with respect to temperature and enforcing the condition $\hat{\eta}(\hat{\F}, 0) = 0$, the isochoric entropy is obtained as:
\begin{equation}
    \hat{\eta}(\hat{\F}, \theta) = \frac{\hat{\eta}_R(\hat{\F})}{\xi_R} \int_0^\theta \frac{g(\vartheta)}{\vartheta} \, d\vartheta.
\end{equation}

Subsequently, using the relation $\hat{\eta} = -(\partial \hat{\Psi} / \partial \theta)$ and integrating once more with respect to temperature, we arrive at the final expression for the isochoric free energy density:
\begin{align}
    \hat{\Psi}(\hat{\F}, \theta) &= \hat{\Psi}_K(\hat{\F}) - \int_0^\theta \hat{\eta}(\hat{\F}, \tau) \, d\tau \nonumber \\
    &= \hat{\Psi}_K(\hat{\F}) - \frac{\hat{\eta}_R(\hat{\F})}{\xi_R} \int_0^\theta \left( \int_0^\tau \frac{g(\vartheta)}{\vartheta} \, d\vartheta \right) d\tau.
\end{align}
Using integration by parts to simplify the double integral, the canonical form is:
\begin{equation} \label{eq:thermo-mech-Kelvin}
    \boxed{
    \hat{\Psi}(\hat{\F}, \theta) = \hat{\Psi}_K(\hat{\F}) - \frac{\hat{\eta}_R(\hat{\F})}{\xi_R} \int_0^\theta g(\vartheta) \left( \frac{\theta}{\vartheta} - 1 \right) \, d\vartheta
    }
\end{equation}
where $\hat{\Psi}_K(\hat{\F})$ represents the stored elastic energy at the Kelvin state (absolute zero).

At the reference temperature $\theta_R$, the energy must match the constitutive model defined by parameters measured at $\theta_R$. This implies that the isochoric energy at $\theta_R$ must be equal to the reference isochoric energy, i.e., $\hat{\Psi}(\hat{\F}, \theta_R) = \hat{\Psi}_R(\hat{\F})$.
The energy can then be expressed relative to the reference temperature $\theta_R$ instead of absolute zero, the expression becomes:
\begin{equation} \label{eq:thero-mech-reference}
    \hat{\Psi}(\hat{\F}, \theta) = \hat{\Psi}_R(\hat{\F})
      - \frac{\hat{\eta}_R(\hat{\F})}{\xi_R} \left(
        \int_0^\theta g(\vartheta) \left( \frac{\theta}{\vartheta} - 1 \right) \, d\vartheta
        -\mathcal{G}(\theta_R)
      \right)\ ,
\end{equation}
where $\mathcal{G}(\theta_R)$ is a constant integral value that depends only on the specific heat function $g(\theta)$:
\begin{equation}
  \mathcal{G}(\theta_R) = \int_0^{\theta_R} g(\vartheta) \left( \frac{\theta_R}{\vartheta} - 1 \right) \, d\vartheta .
\end{equation}



\section*{Melting model}

Additionally, we can consider the physical hypotesis that a specific melting temperature $\theta_M$ exists, such that the material looses its ability to sustain shear stress. This can be mathematically expressed as
\begin{equation} \label{eq:melting_condition}
  \hat{\Psi}(\hat{\F}, \theta_M) = 0\ .
\end{equation}
After substituting Eq. (\ref{eq:melting_condition}) into Eq. (\ref{eq:thero-mech-reference}), we can express the isochoric energy as
\begin{equation}
  \hat\Psi(\F,\theta) = \hat\Psi_R(\F) \frac{\mathcal{G}(\theta_M) - \G(\theta)}{\G(\theta_M) - \G(\theta_R)}\ .
\end{equation}

\vspace{2em}

\textcolor{teal}{Hasta aquí la notación consistente con el artículo de J. Bonet y A. Gil \emph{Finite strain thermoelasticity and the Third Law of thermodynamics}}


$$\sim\sim\sim\boldsymbol{\cdot}\sim\sim\sim$$




\setcounter{equation}{11}
\section{Thermo-mechanical energy}

\begin{equation}
  \Psi(\F, \E_0, \theta, \A) = \bar{\Psi}(J,\theta) + \hat{\Psi}(\F, \E_0, \theta, \A)
\end{equation}

The volumetric contribution is written as

\begin{equation}
  \bar{\Psi}(J,\theta) = \bar{\Psi}_R(J) - \bar{g}(\theta) \theta_R \bar{\eta}_R(J)
\end{equation}

The isochoric contribution adopts the following expression

\begin{equation}
  \hat{\Psi}(\F, \E_0, \theta, \A) = \hat{\Psi}_R(\F, \E_0, \A) - \hat{g}(\theta) \theta_R \hat{\eta}_R(\F, \E_0, \A)
\end{equation}

where the volumetric and isochoric reference entropies are defined as

\begin{equation}
  \bar{\eta}_R = \alpha(J - 1) + \frac{c_v^0}{\bar\gamma} \ , \quad
    \hat{\eta}_R = -\frac{1}{\theta_R} \hat{\Psi}_R(\F, \E_0, \A)
\end{equation}

The purely thermal functions $\bar{g}$ and $\hat{g}$ could be defined as
\begin{equation}
  \bar{g}(\theta) = \frac{1}{\bar\gamma+1}\left(\left(\frac{\theta}{\theta_R}\right)^{\bar\gamma+1}-1\right) \ , \quad
  \hat{g}(\theta) = \frac{1}{\hat\gamma+1}\left(\left(\frac{\theta}{\theta_R}\right)^{\hat\gamma+1}-1\right)
\end{equation}


Introduction of the definition of  $\bar{g}$ and $\hat{g}$ into the definitions of $\bar{\eta}_R$ and $\hat{\eta}_R$, $\bar{\Psi}$ and $\hat{\Psi}$, leads to the following expression for $\Psi$:

\begin{equation}
    \Psi(\F,\E_0,\theta,\A) = \bar{\Psi}_R(J) - \bar{g}(\theta) \theta_R
        \left(\alpha(J - 1) + \frac{c_v^0}{\bar\gamma}\right) +
    \left(1 + \hat{g}(\theta)\right)
        \hat{\Psi}_R(\F,\E_0,\A)
\end{equation}


\textcolor{blue}{
\setcounter{equation}{15}
\begin{equation}
  \hspace{15em}
    \hat{g}(\theta) = \left(\frac{\theta}{\theta_R}\right)^{-\hat\gamma}
\end{equation}
}%
\textcolor{blue}{
\addtocounter{equation}{0}
\begin{equation}
    \Psi(\F,\E_0,\theta,\A) = \bar{\Psi}_R(J) - \bar{g}(\theta) \theta_R
        \left(\alpha(J - 1) + \frac{c_v^0}{\bar\gamma}\right) +
    \hat{g}(\theta)
        \hat{\Psi}_R(\F,\E_0,\A)
\end{equation}
}

The new choice for $\hat{g}$ allows to have a decreasing function with temperature, which is more physically sound for the isochoric part of the energy. In Fig. \ref{fig:experim_fit} we show the fitting of the model to several experimental data from \textcolor{blue}{DIPPEL} and \textcolor{blue}{MOKARRAM}. The left figure shows the specific heat at constant mass as a function of temperature, while the right figure shows stress-stretch curves for different temperatures. In Fig. \ref{fig:g_viscous_effect} we present an in-depth analysis of the experimental data. The left figure shows the scaling factor obtained from the maximum stress, while the right figure shows the experimental stresses divided by the thermal scaling. It can be observed that the maximum stress collapse into a single point, but the curves follow a different path, indicating that a single thermal isochoric function is not able to capture the thermal effects on the viscoelastic response accurately.

\begin{figure}[htb!]
  \centering
  \includegraphics[width=0.49\textwidth]{cv_temp.png}
  \includegraphics[width=0.49\textwidth]{stress_stretch.png}
  \caption{Model fitting with experimental data. \emph{Left:} specific heat at constant volume as a function of temperature. \emph{Right:} stress-stretch curves for different temperatures using the new thermal function \textcolor{blue}{$\hat{g}$}.}
  \label{fig:experim_fit}
\end{figure}


\begin{figure}[htb!]
  \centering
  \includegraphics[width=0.49\textwidth]{g_experim.png}
  \includegraphics[width=0.49\textwidth]{g_viscous.png}
  \caption{In-depth analysis of experimental data. \emph{Left figure:} scaling factor obtained from the maximum stress. \emph{Right figure:} experimental stresses divided by the thermal scaling.}
  \label{fig:g_viscous_effect}
\end{figure}


As a consequence, other choices for $\hat{g}$ may be made. When considering viscoelastic effects, it is interesting to separate the isochoric part of the energy into an elastic and a viscous contribution, each one with its own thermal function, i.e.,

\textcolor{blue}{
\addtocounter{equation}{-1}
\begin{multline}
    \Psi(\F,\E_0,\theta,\A) = \bar{\Psi}_R(J) - \bar{g}(\theta) \theta_R
        \left(\alpha(J - 1) + \frac{c_v^0}{\bar\gamma}\right) \\
    + \hat{g}_e(\theta) \hat{\Psi}_{e,R}(\F,\E_0)
    + \hat{g}_v(\theta) \hat{\Psi}_{v,R}(\F,\E_0,\A)
\end{multline}
}


\subsection{Constraints}

\begin{fleqn}
  \paragraph{Volumetric energy at reference configuration}
  \begin{equation*}
    \bar{\Psi}(\F,\E_0,\theta=\theta_R,\A) = \bar{\Psi}_R(J) \quad \rightarrow \quad \bar{g}(\theta=\theta_R) = 0
  \end{equation*}

  \paragraph{Specific heat at reference configuration}
  \begin{equation*}
    \bar{c}_v(J=1,\theta=\theta_R) = c_v^0 \quad \rightarrow \quad \bar{g}''(\theta=\theta_R) = \bar{\gamma}\theta_R^{-2}
  \end{equation*}

  \paragraph{Isochoric energy at reference configuration}
  \begin{equation*}
    \hat{\Psi}(\F,\E_0,\theta=\theta_R,\A) = \hat{\Psi}_R(\F,\E_0,\A) \quad \rightarrow \quad \hat{g}(\theta=\theta_R) = 1
  \end{equation*}

  \paragraph{Specific heat positivity}
  \begin{equation*}
    c_v \geq 0 \quad \rightarrow \quad \bar{c}_v + \hat{c}_v \geq 0 \quad \rightarrow \quad \dots
  \end{equation*}
\end{fleqn}



\color{gray}

\section{Conservation laws}

\paragraph{Linear momentum}
\begin{equation*}
    \int_\Omega \mathbf{P}(\F, \E_0, \theta) : \nabla v_u \ d\Omega - \int_\Omega \mathbf{f}_0 \cdot v_u \ d\Omega = 0.
\end{equation*}

\paragraph{Gauss law}
\begin{equation*}
    \int_\Omega \D_0(\F, \E_0, \theta) \cdot \nabla v_\varphi \ d\Omega = 0,
\end{equation*}

\paragraph{Energy conservation}
\begin{equation*}
    \frac{d}{dt}(\theta\eta) - \eta\dot\theta - \D_\text{vis}
    + \Div\Q - r_R = 0.
\end{equation*}
with the corresponding weak form
\begin{equation*}
    \int_\Omega \left(\frac{d}{dt}(\theta\eta) -\eta\dot\theta -\D_\text{vis}\right)v_\theta d\Omega
    + \int_\Omega \kappa\nabla\theta\cdot\nabla v_\theta d\Omega
    - \int_\Gamma r_R v_\theta d\Gamma = 0.
\end{equation*}


\begin{equation*}
    \rightarrow K = J^{-1}\H^T\kappa\H
\end{equation*}



\section{Semi-discrete thermal residual}

\begin{multline*}
  \frac{1}{\Delta t}\int_\Omega (\theta_{h,n+1} \eta_{n+1} - \theta_{h,n} \eta_{n}) v_\theta dV
  -\frac{1}{2\Delta t}\int_\Omega (\eta_{n} + \eta_{n+1})(\theta_{h,n+1} - \theta_{h,n}) v_\theta dV \\
  -\frac{1}{2} \int_\Omega (D_{\text{vis},n+1} + D_{\text{vis},n}) v_\theta dV +
  \frac{1}{2} \int_\Omega (\nabla\theta_{h,n+1} + \nabla\theta_{h,n}) k \nabla v_\theta dV = 0
\end{multline*}


\section{Semi-discrete blance of total energy}

Taking the test functions $\{\mathbf{v}_u, v_\phi, v_\theta \}$ as $\{\Delta\mathbf{u}/\Delta t, \Delta\phi/\Delta t, 1\}$

\begin{equation*}
    \mathcal{W}_{u_{n+1/2}}^\text{int} = \frac{1}{\Delta t}\int_\Omega \P_h:\Delta\F dV
\end{equation*}
\begin{equation*}
    \mathcal{W}_{\phi_{n+1/2}}^\text{int} = \frac{1}{\Delta t}\int_\Omega \D_{0,h}\cdot\Delta\D_0 dV
\end{equation*}
\begin{equation*}
    \mathcal{W}_{\theta_{n+1/2}}^\text{int} = \frac{1}{\Delta t}\int_\Omega (\Delta(\eta\theta) + \eta_h\Delta\theta) dV
\end{equation*}





\section{Numerical example}

\subsection{Geometry}

$$
\Omega = 100 \times 100 \times 1
$$


\subsection{Constitutive model}

\paragraph{Neo-hookean parameters}
\begin{align*}
\mu_1   &= 1.0 \\
\mu_2   &= 1.0 \\
\lambda &= 10.0
\end{align*}

\paragraph{Ideal dielectric parameters}
\begin{equation*}
\epsilon = 1.0 
\end{equation*}


\paragraph{Thermal parameters}
\begin{align*}
c_v^0      &= 17.385  \\
\theta_R   &= 293.15  \\
k          &= \lambda + 2(\mu_1+\mu_2) \\
\alpha     &= 22.33k\times10^{-5} \\
\bar\gamma &= 1.0 \\
\hat\gamma &= 1.0
\end{align*}


\subsection{Dirichlet boundary conditions}

\begin{align*}
\mathbf{u} &= \mathbf{0} \ , \quad \text{at} \ \Gamma_{u} \\
\phi    &= 0 \ , \quad \text{at} \ \Gamma_{\phi,\text{bottom}} \\
\phi    &= \phi_D \ , \quad \text{at} \ \Gamma_{\phi,\text{top}}
\end{align*}
\begin{align*}
\phi_D = 0.3\sin\left(\frac{\pi t}{2}\right)
\end{align*}


\subsection{Neumann boundary conditions}

\begin{align*}
r = r_N \ , \quad \text{at} \ \Gamma_{\theta}
\end{align*}
\begin{align*}
r_N = \frac{500}{\pi 15^2}\sin\left(\frac{\pi t}{2}\right)
\end{align*}


\subsection{Results}

$$
\eta_0 = \int_\Omega\frac{c_v^0}{\bar\gamma}dV = 173850.0
$$




\section{Patch test}

\subsection{Geometry}

$$
\Omega = 0.1 \times 0.1 \times 0.1
$$

\begin{align*}
  u_z  = 0    \ &, \quad \text{at} \ \Gamma_\text{bottom} \\
  \phi = 0    \ &, \quad \text{at} \ \Gamma_\text{bottom} \\
  \phi = \alpha t \ &, \quad \text{at} \ \Gamma_\text{top} \\
\end{align*}



\subsection{Energy balance}

\begin{equation*}
    \frac{d}{dt}(\eta\theta) - \eta\dot\theta - D_\text{vis}
    + \Div\Q - r_R = 0
\end{equation*}

\begin{equation*}
    \dot\eta\theta -D_\text{vis} = 0 \ ; \quad \eta=-\frac{\partial\Psi}{\partial\theta}
\end{equation*}

\begin{equation*}
    \dot\eta\theta -D_\text{vis} = 0 \ ; \quad
    \dot\eta =
        -\frac{\partial^2\Psi}{\partial\theta\partial\F}:\dot\F \
        -\frac{\partial^2\Psi}{\partial\theta\partial\E_0}\cdot\dot\E_0 \
        -\frac{\partial^2\Psi}{\partial\theta^2}\dot\theta
\end{equation*}

\begin{equation*}
    \dot\theta = \frac{
        \displaystyle
        \frac{D_\text{vis}}{\theta}
        + \frac{\partial\P}{\partial\theta}:\dot\F
        - \frac{\partial\D_0}{\partial\theta}\cdot\dot\E_0
    }{c_v}
\end{equation*}

$$
\eta_R = -\frac{\partial\Psi}{\partial\theta}(\F=\I, \E=\mathbf{0}, \theta=\theta_R) = \frac{c_v^0}{\bar\gamma}
$$

$$
\int \frac{D_\text{vis}}{c_v\theta} dt
$$

$$
m \, \alpha \, \text{cof}(\C_{e_\alpha}) \otimes \text{cof}(\C_{e_\alpha})
$$

$$
\alpha = \text{tr}\left(\frac{\partial \S_{e_\alpha}'}{\partial \C_{e_\alpha}}\right)
$$


\end{document}
