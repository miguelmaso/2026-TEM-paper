\documentclass[a4paper,12pt]{article}

\usepackage[english]{babel}
\usepackage[utf8]{inputenc}
\usepackage[OT1]{fontenc}
\usepackage{mathtools}
\usepackage{amsmath}
\usepackage{xcolor}

\newcommand{\pder}[2]{\frac{\partial #1}{\partial #2}}
\newcommand{\A}{\mathcal{A}}
\newcommand{\C}{\mathbf{C}}
\newcommand{\D}{\mathbf{D}}
\newcommand{\E}{\mathbf{E}}
\newcommand{\F}{\mathbf{F}}
\newcommand{\I}{\mathbf{I}}
\renewcommand{\H}{\mathbf{H}}  % ¶
\renewcommand{\P}{\mathbf{P}}  % ¶
\renewcommand{\S}{\mathbf{S}}  % §
\newcommand{\Q}{\mathbf{Q}}
\newcommand{\Div}{\text{Div}}
\newcommand{\Eps}{\mathcal{E}}
\newcommand{\0}{\mathbf{0}}


\title{Notes on Thermo-Electro-Visco-Elasticity modelling}
\author{Rogelio Ortigosa, Jesús Martínez, Miguel Masó}
\date{September 5, 2025}

\begin{document}

\addtocounter{equation}{11}

\section{Thermo-mechanical energy}

\begin{equation}
  \Psi(\F, \E_0, \theta, \A) = \bar{\Psi}(J,\theta) + \hat{\Psi}(\F, \E_0, \theta, \A)
\end{equation}

The volumetric contribution is written as

\begin{equation}
  \bar{\Psi}(J,\theta) = \bar{\Psi}_R(J) - \bar{g}(\theta) \theta_R \bar{\eta}_R(J)
\end{equation}

The isochoric contribution adopts the following expression

\begin{equation}
  \hat{\Psi}(\F, \E_0, \theta, \A) = \hat{\Psi}_R(\F, \E_0, \A) - \hat{g}(\theta) \theta_R \hat{\eta}_R(\F, \E_0, \A)
\end{equation}

where the volumetric and isochoric reference entropies are defined as

\begin{equation}
  \bar{\eta}_R = \alpha(J - 1) + \frac{c_v^0}{\bar\gamma} \ , \quad
    \hat{\eta}_R = -\frac{1}{\theta_R} \hat{\Psi}_R(\F, \E_0, \A)
\end{equation}

The purely thermal functions $\bar{g}$ and $\hat{g}$ could be defined as
\begin{equation}
  \bar{g}(\theta) = \frac{1}{\bar\gamma+1}\left(\left(\frac{\theta}{\theta_R}\right)^{\bar\gamma+1}-1\right) \ , \quad
  \hat{g}(\theta) = \frac{1}{\hat\gamma+1}\left(\left(\frac{\theta}{\theta_R}\right)^{\hat\gamma+1}-1\right)
\end{equation}


\textcolor{blue}{
\begin{equation*}
  \hspace{10em}
    \hat{g}(\theta) = \left(\frac{\theta}{\theta_R}\right)^{-\hat\gamma} -1
\end{equation*}
}


Introduction of the definition of  $\bar{g}$ and $\hat{g}$ into the definitions of $\bar{\eta}_R$ and $\hat{\eta}_R$, $\bar{\Psi}$ and $\hat{\Psi}$, leads to the following expression for $\Psi$:

\begin{equation}
    \Psi(\F,\E_0,\theta,\A) = \bar{\Psi}_R(J) - \bar{g}(\theta) \theta_R
        \left(\alpha(J - 1) + \frac{c_v^0}{\bar\gamma}\right) +
    \left(1 + \hat{g}(\theta)\right)
        \hat{\Psi}_R(\F,\E_0,\A)
\end{equation}



\section{Conservation laws}

\paragraph{Linear momentum}
\begin{equation*}
    \int_\Omega \mathbf{P}(\F, \E_0, \theta) : \nabla v_u \ d\Omega - \int_\Omega \mathbf{f}_0 \cdot v_u \ d\Omega = 0.
\end{equation*}

\paragraph{Gauss law}
\begin{equation*}
    \int_\Omega \D_0(\F, \E_0, \theta) \cdot \nabla v_\varphi \ d\Omega = 0,
\end{equation*}

\paragraph{Energy conservation}
\begin{equation*}
    \frac{d}{dt}(\theta\eta) - \eta\dot\theta - \D_\text{vis}
    + \Div\Q - r_R = 0.
\end{equation*}
with the corresponding weak form
\begin{equation*}
    \int_\Omega \left(\frac{d}{dt}(\theta\eta) -\eta\dot\theta -\D_\text{vis}\right)v_\theta d\Omega
    + \int_\Omega \kappa\nabla\theta\cdot\nabla v_\theta d\Omega
    - \int_\Gamma r_R v_\theta d\Gamma = 0.
\end{equation*}


\begin{equation*}
    \rightarrow K = J^{-1}\H^T\kappa\H
\end{equation*}



\section{Semi-discrete thermal residual}

\begin{multline*}
  \frac{1}{\Delta t}\int_\Omega (\theta_{h,n+1} \eta_{n+1} - \theta_{h,n} \eta_{n}) v_\theta dV
  -\frac{1}{2\Delta t}\int_\Omega (\eta_{n} + \eta_{n+1})(\theta_{h,n+1} - \theta_{h,n}) v_\theta dV \\
  -\frac{1}{2} \int_\Omega (D_{\text{vis},n+1} + D_{\text{vis},n}) v_\theta dV +
  \frac{1}{2} \int_\Omega (\nabla\theta_{h,n+1} + \nabla\theta_{h,n}) k \nabla v_\theta dV = 0
\end{multline*}


\section{Semi-discrete blance of total energy}

Taking the test functions $\{\mathbf{v}_u, v_\phi, v_\theta \}$ as $\{\Delta\mathbf{u}/\Delta t, \Delta\phi/\Delta t, 1\}$

\begin{equation*}
    \mathcal{W}_{u_{n+1/2}}^\text{int} = \frac{1}{\Delta t}\int_\Omega \P_h:\Delta\F dV
\end{equation*}
\begin{equation*}
    \mathcal{W}_{\phi_{n+1/2}}^\text{int} = \frac{1}{\Delta t}\int_\Omega \D_{0,h}\cdot\Delta\D_0 dV
\end{equation*}
\begin{equation*}
    \mathcal{W}_{\theta_{n+1/2}}^\text{int} = \frac{1}{\Delta t}\int_\Omega (\Delta(\eta\theta) + \eta_h\Delta\theta) dV
\end{equation*}





\section{Numerical example}

\subsection{Geometry}

$$
\Omega = 100 \times 100 \times 1
$$


\subsection{Constitutive model}

\paragraph{Neo-hookean parameters}
\begin{align*}
\mu_1   &= 1.0 \\
\mu_2   &= 1.0 \\
\lambda &= 10.0
\end{align*}

\paragraph{Ideal dielectric parameters}
\begin{equation*}
\epsilon = 1.0 
\end{equation*}


\paragraph{Thermal parameters}
\begin{align*}
c_v^0      &= 17.385  \\
\theta_R   &= 293.15  \\
k          &= \lambda + 2(\mu_1+\mu_2) \\
\alpha     &= 22.33k\times10^{-5} \\
\bar\gamma &= 1.0 \\
\hat\gamma &= 1.0
\end{align*}


\subsection{Dirichlet boundary conditions}

\begin{align*}
\mathbf{u} &= \mathbf{0} \ , \quad \text{at} \ \Gamma_{u} \\
\phi    &= 0 \ , \quad \text{at} \ \Gamma_{\phi,\text{bottom}} \\
\phi    &= \phi_D \ , \quad \text{at} \ \Gamma_{\phi,\text{top}}
\end{align*}
\begin{align*}
\phi_D = 0.3\sin\left(\frac{\pi t}{2}\right)
\end{align*}


\subsection{Neumann boundary conditions}

\begin{align*}
r = r_N \ , \quad \text{at} \ \Gamma_{\theta}
\end{align*}
\begin{align*}
r_N = \frac{500}{\pi 15^2}\sin\left(\frac{\pi t}{2}\right)
\end{align*}


\subsection{Results}

$$
\eta_0 = \int_\Omega\frac{c_v^0}{\bar\gamma}dV = 173850.0
$$




\section{Patch test}

\subsection{Geometry}

$$
\Omega = 0.1 \times 0.1 \times 0.1
$$

\begin{align*}
  u_z  = 0    \ &, \quad \text{at} \ \Gamma_\text{bottom} \\
  \phi = 0    \ &, \quad \text{at} \ \Gamma_\text{bottom} \\
  \phi = \alpha t \ &, \quad \text{at} \ \Gamma_\text{top} \\
\end{align*}



\subsection{Energy balance}

\begin{equation*}
    \frac{d}{dt}(\eta\theta) - \eta\dot\theta - D_\text{vis}
    + \Div\Q - r_R = 0
\end{equation*}

\begin{equation*}
    \dot\eta\theta -D_\text{vis} = 0 \ ; \quad \eta=-\frac{\partial\Psi}{\partial\theta}
\end{equation*}

\begin{equation*}
    \dot\eta\theta -D_\text{vis} = 0 \ ; \quad
    \dot\eta =
        -\frac{\partial^2\Psi}{\partial\theta\partial\F}:\dot\F \
        -\frac{\partial^2\Psi}{\partial\theta\partial\E_0}\cdot\dot\E_0 \
        -\frac{\partial^2\Psi}{\partial\theta^2}\dot\theta
\end{equation*}

\begin{equation*}
    \dot\theta = \frac{
        \displaystyle
        \frac{D_\text{vis}}{\theta}
        + \frac{\partial\P}{\partial\theta}:\dot\F
        - \frac{\partial\D_0}{\partial\theta}\cdot\dot\E_0
    }{c_v}
\end{equation*}

$$
\eta_R = -\frac{\partial\Psi}{\partial\theta}(\F=\I, \E=\mathbf{0}, \theta=\theta_R) = \frac{c_v^0}{\bar\gamma}
$$

$$
\int \frac{D_\text{vis}}{c_v\theta} dt
$$

$$
m \, \alpha \, \text{cof}(\C_{e_\alpha}) \otimes \text{cof}(\C_{e_\alpha})
$$

$$
\alpha = \text{tr}\left(\frac{\partial \S_{e_\alpha}'}{\partial \C_{e_\alpha}}\right)
$$


\end{document}
